\input{../texheader/ebook}

\title{A New Kind of Science}
\author{Stephen Wolfram\\
{\tiny перевод Dmitry Ponyatov \email{dponyatov@gmail.com}}}

\begin{document}
\maketitle
\tableofcontents\secdown

\secly{Preface}

\secrel{The Foundations for a New Kind of Science}\secdown
\secrel{An Outline of Basic Ideas}
\secrel{Relations to Other Areas}
\secrel{Some Past Initiatives}
\secrel{The Personal Story of the Science in This Book}
\secup

\secrel{The Crucial Experiment}\secdown
\secrel{How Do Simple Programs Behave?}
\secrel{The Need for a New Intuition}
\secrel{Why These Discoveries Were Not Made Before}
\secup

\secrel{The World of Simple Programs}\secdown
\secrel{The Search for General Features}
\secrel{More Cellular Automata}
\secrel{Mobile Automata}
\secrel{Turing Machines}
\secrel{Substitution Systems}
\secrel{Sequential Substitution Systems}
\secrel{Tag Systems}
\secrel{Cyclic Tag Systems}
\secrel{Register Machines}
\secrel{Symbolic Systems}
\secrel{Some Conclusions}
\secrel{How the Discoveries in This Chapter Were Made}
\secup

\secrel{Systems Based on Numbers}\secdown
\secrel{The Notion of Numbers}
\secrel{Elementary Arithmetic}
\secrel{Recursive Sequences}
\secrel{The Sequence of Primes}
\secrel{Mathematical Constants}
\secrel{Mathematical Functions}
\secrel{Iterated Maps and the Chaos Phenomenon}
\secrel{Continuous Cellular Automata}
\secrel{Partial Differential Equations}
\secrel{Continuous Versus Discrete Systems}
\secup

\secrel{Two Dimensions and Beyond}\secdown
\secrel{Introduction}
\secrel{Cellular Automata}
\secrel{Turing Machines}
\secrel{Substitution Systems and Fractals}
\secrel{Network Systems}
\secrel{Multiway Systems}
\secrel{Systems Based on Constraints}
\secup

\secrel{Starting from Randomness}\secdown
\secrel{The Emergence of Order}
\secrel{Four Classes of Behavior}
\secrel{Sensitivity to Initial Conditions}
\secrel{Systems of Limited Size and Class 2 Behavior}
\secrel{Randomness in Class 3 Systems}
\secrel{Special Initial Conditions}
\secrel{The Notion of Attractors}
\secrel{Structures in Class 4 Systems}
\secup

\secrel{Mechanisms in Programs and Nature}\secdown
\secrel{Universality of Behavior}
\secrel{Three Mechanisms for Randomness}
\secrel{Randomness from the Environment}
\secrel{Chaos Theory and Randomness from Initial Conditions}
\secrel{The Intrinsic Generation of Randomness}
\secrel{The Phenomenon of Continuity}
\secrel{Origins of Discreteness}
\secrel{The Problem of Satisfying Constraints}
\secrel{Origins of Simple Behavior}
\secup

\secrel{Implications for Everyday Systems}\secdown
\secrel{Issues of Modelling}
\secrel{The Growth of Crystals}
\secrel{The Breaking of Materials}
\secrel{Fluid Flow}
\secrel{Fundamental Issues in Biology}
\secrel{Growth of Plants and Animals}
\secrel{Biological Pigmentation Patterns}
\secrel{Financial Systems}
\secup

\secrel{Fundamental Physics}\secdown
\secrel{The Problems of Physics}
\secrel{The Notion of Reversibility}
\secrel{Irreversibility and the Second Law of Thermodynamics}
\secrel{Conserved Quantities and Continuum Phenomena}
\secrel{Ultimate Models for the Universe}
\secrel{The Nature of Space}
\secrel{Space as a Network}
\secrel{The Relationship of Space and Time}
\secrel{Time and Causal Networks}
\secrel{The Sequencing of Events in the Universe}
\secrel{Uniqueness and Branching in Time}
\secrel{Evolution of Networks}
\secrel{Space, Time and Relativity}
\secrel{Elementary Particles}
\secrel{The Phenomenon of Gravity}
\secrel{Quantum Phenomena}
\secup

\secrel{Processes of Perception and Analysis}\secdown
\secrel{Introduction}
\secrel{What Perception and Analysis Do}
\secrel{Defining the Notion of Randomness}
\secrel{Defining Complexity}
\secrel{Data Compression}
\secrel{Irreversible Data Compression}
\secrel{Visual Perception}
\secrel{Auditory Perception}
\secrel{Statistical Analysis}
\secrel{Cryptography and Cryptanalysis}
\secrel{Traditional Mathematics and Mathematical Formulas}
\secrel{Human Thinking}
\secrel{Higher Forms of Perception and Analysis}
\secup

\secrel{The Notion of Computation}\secdown
\secrel{Computation as a Framework}
\secrel{Computations in Cellular Automata}
\secrel{The Phenomenon of Universality}
\secrel{A Universal Cellular Automaton}
\secrel{Emulating Other Systems with Cellular Automata}
\secrel{Emulating Cellular Automata with Other Systems}
\secrel{Implications of Universality}
\secrel{The Rule 110 Cellular Automaton}
\secrel{The Significance of Universality in Rule }
\secrel{Class 4 Behavior and Universality}
\secrel{The Threshold of Universality in Cellular Automata}
\secrel{Universality in Turing Machines and Other Systems}
\secup

\secrel{The Principle of Computational Equivalence}\secdown
\secrel{Basic Framework}
\secrel{Outline of the Principle}
\secrel{The Content of the Principle}
\secrel{The Validity of the Principle}
\secrel{Explaining the Phenomenon of Complexity}
\secrel{Computational Irreducibility}
\secrel{The Phenomenon of Free Will}
\secrel{Undecidability and Intractability}
\secrel{Implications for Mathematics and Its Foundations}
\secrel{Intelligence in the Universe}
\secrel{Implications for Technology}
\secrel{Historical Perspectives}
\secup

\secly{Notes}

\end{document}